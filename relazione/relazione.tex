\documentclass{article}
\usepackage[utf8]{inputenc}
\usepackage[italian]{babel}
\usepackage{graphicx}
\usepackage[table]{xcolor}

\title{Progetto Palestra - Sito Web}
\author{Armando Moda Scarati - 2082864, Granziero Luca - 2075512, Alessandro, Marco}
\date{\today}

\begin{document}

\maketitle

\section{Introduzione}
Questo documento presenta il progetto per la realizzazione del sito web di un’associazione che gestisce diverse palestre dotate di sala pesi, corsi fitness e un’area spa. Il sito prevede una struttura completa, con pagine dedicate agli abbonamenti, ai corsi, ai servizi della spa, alla sezione Contatti e una pagina Home che offra una panoramica introduttiva chiara.

\section{Analisi del target}
Il sito web è destinato a due principali categorie di utenti: gli utenti finali, che sono iscritti alla palestra e utilizzano i servizi offerti, e i personal trainer (PT), che forniscono servizi di coaching personalizzato. Gli utenti finali possono essere persone interessate al fitness, al benessere e ai servizi spa, mentre i PT sono professionisti che gestiscono sessioni di allenamento e piani alimentari personalizzati.

\section{Struttura del Sito}
Il sito è composto da: \\
\\
\textbf{Pagine html:}
\begin{itemize}
    \item \texttt{home.html}: Pagina principale con logo e contenuti in italiano.
    \item \texttt{corsi.html:} Pagina che espone i vari corsi disponibili.
    \item \texttt{abbonamenti.html:} Pagina che raffigura i vari abbonamenti offerti.
    \item \texttt{palestra.html:} Pagina di intermissione tra home e i servizi che la palestra offre, ovvero la sala pesi, i corsi e i pt/coaching.

    \item \texttt{error404.html:} Pagina che mostra un messaggio quando la pagina richiesta non esiste.
    \item \texttt{contattaci.html:} Pagina che permette agli utenti di inviare richieste o messaggi all'associazione.
    
\end{itemize} 
    \textbf{Pagine css:}
\begin{itemize}
    \item \texttt{style.css}: Foglio di stile per il design base.
\end{itemize}
 \textbf{Pagine javascript:}
\begin{itemize}
    \item 
\end{itemize}
\textbf{Pagine php:}
\begin{itemize}
    \item 
\end{itemize}

\section{Tecnologie Utilizzate}
\begin{itemize}
    \item \textbf{HTML5}: Utilizzato per definire la struttura semantica delle pagine web, fornendo elementi come header, nav, section e footer per una migliore accessibilità e SEO.
    \item \textbf{CSS}: Impiegato per lo styling e il layout delle pagine, garantendo un design responsive e adattabile a diversi dispositivi.
    \item \textbf{JavaScript}: Integrato per aggiungere interattività e animazioni dinamiche, migliorando l'esperienza utente senza ricaricare la pagina.
    \item \textbf{PHP}: Utilizzato per generare contenuti dinamici lato server, come la gestione di sessioni utente e l'interazione con database per funzionalità avanzate.
    \item \textbf{SQL}: Linguaggio utilizzato per interrogare e gestire database relazionali, essenziale per archiviare e recuperare dati utente, abbonamenti e altre informazioni del sito.
    \item \textbf{LaTeX}: Adottato per la stesura di questa relazione, permettendo la creazione di documenti di alta qualità con formattazione professionale.
\end{itemize}
Infine una cartella immagini con all'interno:
\begin{itemize}
    \item immagine logo palestra.
\end{itemize}
\section{Attori presenti nel sito}
All'interno del sito troviamo tre attori:\\
\texttt{utente/user} \\
\texttt{utente personal trainerr} \\
\texttt{admin/amministratore}. \\
\noindent
\\
Le credenziali necessarie per il login dei rispettivi sono: \\
\begin{center}
\begin{tabular}{|c|c|c|}
\hline
\rowcolor{lightgray}\textbf{Ruolo} & \textbf{Username} & \textbf{Password} \\
\hline
USER  & user  & user  \\
\hline
PT & pt & pt \\
\hline
ADMIN & admin & admin \\
\hline
\end{tabular}
\end{center} 
\bigskip
Di seguito sono riportate le funzionalità disponibili per i ruoli User e Admin:
\subsection{USER}
\begin{itemize}
    \item Gestione semplice della disdetta dell’abbonamento (valutare se mantenere).
    \item Visualizzazione della data di attivazione, scadenza e dello stato attuale dell’abbonamento (attivo, in scadenza, scaduto).
    \item Visualizzazione orari, posti disponibili e istruttori dei corsi a cui    l’utente è iscritto.
    \item Calendario dei corsi a cui si è iscritti e degli appuntamenti con il personal trainer.
    \item Visualizzazione della scheda di allenamento personalizzata (se presente personal trainer).
    \item Visualizzazione del piano alimentare personalizzato (se presente personal trainer).
    \item Possibilità di lasciare recensioni (da valutare per questioni di sicurezza).
    \item Gestione del proprio account (cancellazione account, modifica dei dati).
\end{itemize}

\subsection{PT}
\begin{itemize}
    \item Profilo pubblico, visualizzabile da chiunque.
    \item Calendario con le disponibilità e le fasce orarie già occupate da altri utenti.
    \item Possibilità di caricare scheda di allenamento e piano alimentare per i clienti che lo richiedono.
    \item Lista dei clienti (confermata), con possibilità di chat con ogni cliente non in tempo reale (da valutare).
\end{itemize}

\subsection{ADMIN}
\begin{itemize}
    \item Possibilità di creare account personal trainer o convertire un account utente in account PT.
    \item Visualizzazione e gestione di tutti gli abbonamenti attivi, sospesi o scaduti.
    \item Gestione del profilo utente (schede, corsi, prenotazioni, PT assegnati, scadenze, ecc.).
    \item Gestione degli istruttori/personal trainer: calendario, disponibilità, schede clienti, retribuzione (possibile).
    \item Dashboard con panoramica di ricavi, abbonamenti attivi, nuovi iscritti e lista degli iscritti attivi.
    \item Statistiche sull’utilizzo della palestra, corsi più frequentati e tassi di retention (possibile).
    \item Cancellazione account (sia utente che PT) tramite dashboard.
    \item Possibilità di gestione dei corsi (modifica della capienza e degli orari).
\end{itemize}

\section{Accessibilità}
\subsection{Separazione della struttura, presentazione e contenuto}
La separazione tra struttura (HTML), presentazione (CSS) e contenuto garantisce che il sito sia accessibile a tutti gli utenti, inclusi quelli che utilizzano screen reader o dispositivi assistive. L'HTML semantico facilita la navigazione e la comprensione del contenuto.

\subsection{Palette dei colori e contrasti}
La palette dei colori è stata scelta per garantire un elevato contrasto tra testi e sfondi, rispettando le linee guida WCAG. Ad esempio, testi chiari su sfondi scuri per una migliore leggibilità, evitando combinazioni che potrebbero causare difficoltà a utenti con disabilità visive.

\subsection{Lingue straniere e abbreviazioni}
Il sito supporta contenuti in italiano come lingua principale, ma è progettato per essere facilmente adattabile a lingue straniere attraverso l'uso di attributi lang e meccanismi di internazionalizzazione, permettendo una migliore accessibilità per utenti non italofoni. Inoltre, le abbreviazioni utilizzate nel sito, come "PT" per "Personal Trainer", sono spiegate o espanse dove necessario per garantire chiarezza e facilitare la comprensione da parte di tutti gli utenti, inclusi quelli con disabilità cognitive o che utilizzano tecnologie assistive.

\section{Divisione dei compiti}
Vengono qui esposti i vari compiti svolti da ciascun membro del team di progetto: \\
\noindent
\\
\textbf{Armando Moda Scarati}
\begin{itemize}
    \item Creazione delle e gestione pagine: \texttt{abbonamenti.html}, \texttt{palestra.html} 
    \item Gestione in css delle stesse 
\end{itemize}
\textbf{Alessandro}
\begin{itemize}
    \item Creazione e gestione delle pagine: \texttt{home.html}
    \item Gestione in css delle stesse 
\end{itemize}
\textbf{Granziero Luca}
\begin{itemize}
    \item Creazione e gestione delle pagine: \texttt{corsi.html} \texttt{error404.html}
    \item Gestione in css delle stesse
    \item Stesura e aggiornamento della relazione 
\end{itemize}
\textbf{Marco}
\begin{itemize}
    \item Gestione della pagina: \texttt{abbonamenti.html} 
    \item Creazione delle pagina: \texttt{contattaci.html}
    \item Gestione in css delle stesse
\end{itemize}

\section{Conclusioni}
Il progetto fornisce una base per un sito web di palestra, pronto per ulteriori sviluppi.

\end{document}
