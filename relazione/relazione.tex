\documentclass[a4paper,12pt]{article}
\usepackage[utf8]{inputenc}
\usepackage[italian]{babel}
\usepackage{graphicx}
\usepackage[table]{xcolor}
\usepackage{hyperref}
\usepackage[a4paper, margin=2.5cm]{geometry}
\usepackage[T1]{fontenc}
\usepackage{lmodern}
\usepackage{textcomp}
\begin{document}

\begin{titlepage}
    \centering
    \vspace*{2cm}
%    \includegraphics[width=0.5\textwidth]{Logo_Università_Padova.svg.png} 
    
    \vspace{1.5cm}
    {\Huge \textbf{Tempio di Apollo - Sito Web} \par}
    \vspace{1cm}
    
    {\Large Armando Moda Scarati - 2082864 \par}
    {\Large Granziero Luca - 2075512 \par}
    {\Large Alessandro Morabito - 2137991 \par}
    {\Large Marco Beltrame - 2111021 \par}
    
    \vspace{1cm}
    {\large \today \par}
    
    \vspace{2cm}
    \textbf{Credenziali account del sito:}
    \vspace{0.5cm}
    
    \begin{center}
    \begin{tabular}{|c|c|c|}
    \hline
    \rowcolor{lightgray}\textbf{Ruolo} & \textbf{Username} & \textbf{Password} \\
    \hline
    USER  & user  & user  \\
    \hline
    PT & pt & pt \\
    \hline
    ADMIN & admin & admin \\
    \hline
    \end{tabular}
    \end{center}
    
    \vfill
\end{titlepage}

\tableofcontents
\newpage

\section{Introduzione}
Questo documento presenta il progetto per la realizzazione del sito web di una palestra che offre i seguenti servizi: sala pesi, corsi fitness e piani personalizzati da un personal trainer. Il sito prevede pagine che offrono una panoramica introduttiva chiara sui servizi che offre.
Prevede inoltre le seguenti tipologie di area personale: per i clienti, per gli istruttori, personal trainer, e per gli admin.  

\section{Analisi del target}
Il sito web è destinato:
\begin{itemize}
    \item ai potenziali clienti che, non ancora registrati, possono informarsi sui servizi che la palestra offre.
    \item ai clienti che tramite un area personale, possono visualizzare le proprie informazioni. 
    \item agli istruttori, la cui area personale è come quella dei clienti, ma possiedono un profilo pubblico
    \item ai personal trainer, tutti anche istruttori, che possono caricare per i propri clienti una scheda tecnica in formato pdf, di grandezza minore di 1MiB
    \item agli amministratori, che possono visualizzare la lista degli utenti iscritti e modificare le loro informazioni
\end{itemize}

\section{Struttura del Sito}
Il sito è composto da: \\
\\
\textbf{Pagine html:}
\begin{itemize}
    \item \texttt{home.html}: Pagina principale che introduce la palestra: mostrando servizi, sedi e recensioni di alcuni clienti.
    \item \texttt{corsi.html:} Pagina che espone i vari corsi disponibili.
    \item \texttt{abbonamenti.html:} Pagina che raffigura i vari abbonamenti offerti.
    \item \texttt{palestra.html:} Pagina di intermissione tra home e i servizi che la palestra offre, ovvero la sala pesi, i corsi e i pt/coaching.
    \item \texttt{error404.html:} Pagina che mostra un messaggio quando la pagina richiesta non esiste.
    \item \texttt{error403.html:} Pagina che mostra un messaggio quando non si hanno i permessi necessari per accedere a quella pagina.
    \item \texttt{contattaci.html:} Pagina che permette agli utenti di inviare richieste o messaggi all'associazione.
    \item \texttt{pt.html:} Pagina che permette agli admin di trovare un preciso PT registrato.
    \item \texttt{cerca-pt.html:} Pagina che permette agli utenti di cercare tra i vari PT quello che più gli aggrada.
    \item \texttt{cerca-utenti.html:} Pagina che permette agli admin di trovare un preciso utente registrato
    \item \texttt{registrati.html:} Pagina che permette agli utenti di registrarsi al sito.
    \item \texttt{login.html:} Pagina che permette agli utenti di accedere alla loro area riservata.
    \item \texttt{utente-admin.html:} Pagina di area personale dell'admin.    
    \item \texttt{utente-pt.html:} Pagina di area personale del Personal Trainer.  
    \item \texttt{utente-semplice.html:} Pagina di area personale dell'utente.
    \item \texttt{salapesi.html:} Pagina che parla delle varie attrezzature e macchinari presenti nella palestra.
\end{itemize} 

\textbf{Pagine css:}
\begin{itemize}
    \item \texttt{style.css}: Foglio di stile per il design base.
    \item \texttt{mini.css}: Foglio di stile per il design mobile. 
    \item \texttt{print.css} Foglio di stile per il design da stampante.
\end{itemize}

\textbf{Pagine javascript:}
\begin{itemize}
    \item \texttt{torna-su.js:} abilità il bottone per tornare in cima alla pagina riposto in basso a destra di ogni pagina.
    \item \texttt{selettoreAbbonamenti.js:}  
    \item \texttt{caricaLogin:}  
    \item \texttt{caricaMessaggio.js:}  
    \item \texttt{caricaRegistrazione.js:}  
    \item \texttt{caricaScheda.js:}  
    \item \texttt{dashboard.js:}
    \item \texttt{formLista.js:}
    \item \texttt{mostraPassword.js:}

\end{itemize}

\textbf{Pagine php:}
\begin{itemize}
    \item \texttt{abbonamenti.php:} 
    \item \texttt{area-personale.php:} 
    \item \texttt{cerca-istruttore.php:} 
    \item \texttt{contattaci.php:} 
    \item \texttt{corsi.php:} 
    \item \texttt{home.php:} 
    \item \texttt{lista-clienti-pt.php:} 
    \item \texttt{lista-clienti.php:} 
    \item \texttt{lista-pt.php:} 
    \item \texttt{lista-utenti.php:} 
    \item \texttt{login.php:} 
    \item \texttt{logout.php:}
    \item \texttt{palestra.php:} 
    \item \texttt{profilo-istruttore.php:} 
    \item \texttt{registrati.php:} 
    \item \texttt{salapesi.php:}
    \item \texttt{403.php:} 
    \item \texttt{404.php:} 
    \item \texttt{upload-messaggio.php:} 
    \item \texttt{processa-login.php:}
    \item \texttt{processa-registrazione.php:} 
    \item \texttt{profilo-istruttore.php:} 
    \item \texttt{upload-scheda.php:} 
    \item \texttt{header.php:}
    \item \texttt{palestra.php:} 
    \item \texttt{utente-admin.php:} 
    \item \texttt{utente-pt.php:} 
    \item \texttt{utente-semplice.php:}
    
\end{itemize}

\textbf{Database:}
\begin{itemize}
    \item \texttt{db.sql:}
\end{itemize}

\textbf{Infine una cartella immagini con all'interno:}
\begin{itemize}
    \item varie immagini utilizzate all'interno del sito.
    \item icon per rendere più apprezzabili alcune liste
    \item immagini dei profili dei vari istruttori e personal trainer
\end{itemize}

\section{Tecnologie Utilizzate}
\begin{itemize}
    \item \textbf{HTML5}: Utilizzato per definire la struttura semantica delle pagine web, fornendo elementi come header, nav, section e footer per una migliore accessibilità e SEO.
    \item \textbf{CSS}: Impiegato per lo styling e il layout delle pagine, garantendo un design responsive e adattabile a diversi dispositivi.
    \item \textbf{JavaScript}: Integrato per aggiungere interattività e animazioni dinamiche, migliorando l'esperienza utente senza ricaricare la pagina.
    \item \textbf{PHP}: Utilizzato per generare contenuti dinamici lato server, come la gestione di sessioni utente e l'interazione con database per funzionalità avanzate.
    \item \textbf{SQL}: Linguaggio utilizzato per interrogare e gestire database relazionali, essenziale per archiviare e recuperare dati utente, abbonamenti e altre informazioni del sito.
    \item \textbf{LaTeX}: Adottato per la stesura di questa relazione, permettendo la creazione di documenti di alta qualità con formattazione professionale.
    \item \textbf{LLM}: Utilizzati per la generazione di immagini come: il logo, le immagini di sfondo presenti nelle pagine di error403 e error404 e le immagini dei vari PT presenti. Usato anche per la popolazione di alcune descrizioni (PT) 
\end{itemize}

\section{Utenti presenti nel sito}
All'interno del sito troviamo tre tipi di utenti:\\
\texttt{utente/user} \\
\texttt{istruttori e personal trainer} \\
\texttt{admin/amministratore}. \\
\noindent
\\

Di seguito sono riportate le funzionalità disponibili per i ruoli User e Admin:
\subsection{USER}
\begin{itemize}
    \item Gestione semplice della disdetta dell’abbonamento (valutare se mantenere).
    \item Visualizzazione della data di attivazione, scadenza e dello stato attuale dell’abbonamento (attivo, in scadenza, scaduto).
    \item Visualizzazione orari, posti disponibili e istruttori dei corsi a cui    l’utente è iscritto.
    \item Calendario dei corsi a cui si è iscritti e degli appuntamenti con il personal trainer.
    \item Visualizzazione della scheda di allenamento personalizzata (se presente personal trainer).
    \item Visualizzazione del piano alimentare personalizzato (se presente personal trainer).
    \item Possibilità di lasciare recensioni (da valutare per questioni di sicurezza).
    \item Gestione del proprio account (cancellazione account, modifica dei dati).
\end{itemize}

\subsection{PT}
\begin{itemize}
    \item Profilo pubblico, visualizzabile da chiunque.
    \item Calendario con le disponibilità e le fasce orarie già occupate da altri utenti.
    \item Possibilità di caricare scheda di allenamento e piano alimentare per i clienti che lo richiedono.
    \item Lista dei clienti (confermata), con possibilità di chat con ogni cliente non in tempo reale (da valutare).
\end{itemize}

\subsection{ADMIN}
\begin{itemize}
    \item Possibilità di creare account personal trainer o convertire un account utente in account PT.
    \item Visualizzazione e gestione di tutti gli abbonamenti attivi, sospesi o scaduti.
    \item Gestione del profilo utente (schede, corsi, prenotazioni, PT assegnati, scadenze, ecc.).
    \item Gestione degli istruttori/personal trainer: calendario, disponibilità, schede clienti, retribuzione (possibile).
    \item Dashboard con panoramica di ricavi, abbonamenti attivi, nuovi iscritti e lista degli iscritti attivi.
    \item Statistiche sull’utilizzo della palestra, corsi più frequentati e tassi di retention (possibile).
    \item Cancellazione account (sia utente che PT) tramite dashboard.
    \item Possibilità di gestione dei corsi (modifica della capienza e degli orari).
\end{itemize}


\section{Struttura del sito}
Alla fine del capitolo è presente lo schema della struttura del sito, visualizzata in base ai link che può seguire l'utente. I breadcrumb sono sempre relativi al primo livello (quello della homepage), ad eccezione delle 3 possibili aree personali, una per tipologia di utenza. L'utente visualizza sempre lo stesso indirizzo sulla barra del browser, ma la schermata varia in base al suo ruolo.

Il motivo per cui abbiamo deciso questa struttura (con collegamenti circolari) è che volevamo permettere all'utente di raggiungere pagine utili direttamente dalla sua area personale.

Si fa notare che "Logout" non è effettivamente una pagina, ma semplicemente un link che fa effettuare l'operazione di logout e reindirizza l'utente, ora non autenticato, alla homepage del sito. Inoltre, "Area Personale" e "Logout" prendono il posto di "Login" e "Registrati" nel menu principale, una volta effettuato il login.

Si fa presente, inoltre, che non sono presenti le pagine php pubbliche che permettono di effettuare le varie operazioni di cui si è parlato nel capitolo precedente. Le pagine invece che abbiamo ritenuto non dover essere accessibili (come quelle delle aree personali della singola tipologia di utente) sono state inserite all'interno della cartella \texttt{internal/}, non visualizzabile pubblicamente in quanto esterna a \texttt{public\textunderscore html/}.

\begin{center}
	\includegraphics[width=.8\linewidth]{struttura_sito.png}
\end{center}

\section{Progettazione}

\section{Accessibilità}
\subsection{Separazione della struttura, presentazione e contenuto}
La separazione tra struttura (HTML), presentazione (CSS) e contenuto garantisce che il sito sia accessibile a tutti gli utenti, inclusi quelli che utilizzano screen reader o dispositivi assistive. L'HTML semantico facilita la navigazione e la comprensione del contenuto.

\subsection{Palette dei colori e contrasti}
La palette dei colori è stata scelta per garantire un elevato contrasto tra testi e sfondi, rispettando le linee guida WCAG. Ad esempio, testi chiari su sfondi scuri per una migliore leggibilità, evitando combinazioni che potrebbero causare difficoltà a utenti con disabilità visive.

\bigskip
\noindent
\textbf{Palette dei colori utilizzata:}

\begin{center}
\begin{tabular}{|c|c|c|}
\hline
\rowcolor{lightgray} \textbf{Colore} & \textbf{Codice HEX} & \textbf{Utilizzato per} \\
\hline
\cellcolor[HTML]{c0ebfb}\hspace{2cm} & \texttt{\#c0ebfb}  & Links non visitati \\
\hline
\cellcolor[HTML]{965cfd}\hspace{2cm} & \texttt{\#965cfd}  & Links non visitati \\
\hline
\cellcolor[HTML]{ffee55}\hspace{2cm} & \texttt{\#ffee55}  & Testo \\
\hline
\cellcolor[HTML]{ffcc00}\hspace{2cm} & \texttt{\#ffcc00}  & Testo \\
\hline
\cellcolor[HTML]{ee8800}\hspace{2cm} & \texttt{\#ee8800}  & Testo \\
\hline
\cellcolor[HTML]{333322}\hspace{2cm} & \texttt{\#333322} &  Sfondo per tabelle \\
\hline 
\cellcolor[HTML]{181211}\hspace{2cm} & \texttt{\#181211}  & Sfondo \\
\hline
\cellcolor[HTML]{111111}\hspace{2cm} & \texttt{\#111111}  & Sfondo header\\
\hline
\cellcolor[HTML]{dddddd}\hspace{2cm} & \texttt{\#dddddd}  & Testi secondari \\
\hline
\end{tabular}
\end{center}
\bigskip 
I colori principali utilizzati nel sito sono riportati nella tabella; eventuali altri colori, impiegati in modo occasionale, sono stati comunque scelti e verificati nel rispetto dei principi di accessibilità.


\subsection{Lingue straniere}
Il sito supporta contenuti in italiano come lingua principale, ma è progettato per essere facilmente adattabile a lingue straniere attraverso l'uso di attributi lang e meccanismi di internazionalizzazione, permettendo una migliore accessibilità per utenti non italofoni.

\section{Divisione dei compiti}
Vengono qui esposti i vari compiti svolti da ciascun membro del team di progetto: \\
\noindent
\\
\textbf{Armando Moda Scarati}
\begin{itemize}
    \item Creazione delle e gestione pagine: \texttt{abbonamenti.html}, \texttt{palestra.html},  \texttt{utente-admin.html},  \texttt{utente-pt.html},  \texttt{utente-semplice.html}
    \item Gestione in css delle stesse 
\end{itemize}
\textbf{Alessandro}
\begin{itemize}
    \item Creazione e gestione delle pagine: \texttt{home.html},  \texttt{lista-utenti.html},  \texttt{pt.html},  \texttt{cerca-pt.html}
    \item Gestione in css delle stesse 
\end{itemize}
\textbf{Granziero Luca}
\begin{itemize}
    \item Creazione e gestione delle pagine: \texttt{corsi.html} \texttt{error404.html}, \textit{erro403.html}
    \item Gestione in css delle stesse
    \item Creazione delle pagine js: toggle-descrizione-orari.js 
    \item Stesura e aggiornamento della relazione 
\end{itemize}
\textbf{Marco}
\begin{itemize}
    \item Gestione della pagine: \texttt{abbonamenti.html}, \texttt{contattaci.html},  \texttt{login.html},  \texttt{registrati.html}
    \item Gestione in css delle stesse
    \item Crezione prima versione del database
\end{itemize}

\section{Conclusioni e sviluppi futuri}
Il progetto costituisce una base strutturata per lo sviluppo di un sito web per una palestra, progettato in modo da consentire agevolmente future espansioni e l’implementazione di ulteriori funzionalità, quali ad esempio:
\end{document}
