\documentclass{article}
\usepackage[utf8]{inputenc}
\usepackage[italian]{babel}
\usepackage{graphicx}
\usepackage[table]{xcolor}

\title{Progetto Palestra - Sito Web}
\author{Armando Moda Scarati - 2082864, Granziero Luca - 2075512, Alessandro, Marco}
\date{\today}

\begin{document}

\maketitle

\section{Introduzione}
Questo documento presenta il progetto per la realizzazione del sito web di un’associazione che gestisce diverse palestre dotate di sala pesi, corsi fitness e un’area spa. Il sito prevede una struttura completa, con pagine dedicate agli abbonamenti, ai corsi, ai servizi della spa, alla sezione Contatti e una pagina Home che offra una panoramica introduttiva chiara.

\section{Struttura del Sito}
Il sito è composto da: \\
\\
\textbf{Pagine html:}
\begin{itemize}
    \item \texttt{home.html}: Pagina principale con logo e contenuti in italiano.
    \item \texttt{corsi.html:} Pagina che espone i vari corsi disponibili.
    \item \texttt{abbonamenti.html:} Pagina che raffigura i vari abbonamenti offerti.
    \item \texttt{palestra.html:} Pagina di intermissione tra home e i servizi che la palestra offre, ovvero la sala pesi, i corsi e i pt/coaching.
    \item \texttt{spa.html:} Pagina che descrive i vari servizi inclusi della spa.
    \item \texttt{error404.html:} Pagina che mostra un messaggio quando la pagina richiesta non esiste.
    \item \texttt{contattaci.html:} Pagina che permette agli utenti di inviare richieste o messaggi all'associazione.
    
\end{itemize} 
    \textbf{Pagine css:}
\begin{itemize}
    \item \texttt{style.css}: Foglio di stile per il design base.
\end{itemize}
 \textbf{Pagine javascript:}
\begin{itemize}
    \item 
\end{itemize}
\textbf{Pagine php:}
\begin{itemize}
    \item 
\end{itemize}

\section{Tecnologie Utilizzate}
\begin{itemize}
    \item HTML5 per la struttura.
    \item CSS per lo styling.
    \item LaTeX per questa relazione.
    \item Javascript per le animazioni.
    \item PHP per generare pagine web dinamiche.
\end{itemize}
Infine una cartella immagini con all'interno:
\begin{itemize}
    \item immagine logo palestra.
   
\end{itemize}
\section{Attori presenti nel sito}
All'interno del sito troviamo due attori: \texttt{utente/user} e \texttt{admin/amministratore}. \\
\noindent
\\
Le credenziali necessarie per il login dei rispettivi sono: \\
\begin{center}
\begin{tabular}{|c|c|c|}
\hline
\rowcolor{lightgray}\textbf{Ruolo} & \textbf{Username} & \textbf{Password} \\
\hline
USER  & user  & user  \\
\hline
ADMIN & admin & admin \\
\hline
\end{tabular}
\end{center} 
\bigskip
Di seguito sono riportate le funzionalità disponibili per i ruoli User e Admin:
\subsection{USER}
\begin{itemize}
    \item 
\end{itemize}

\subsection{ADMIN}
\begin{itemize}
    \item 
\end{itemize}



\section{Divisione dei compiti}
Vengono qui esposti i vari compiti svolti da ciascun membro del team di progetto: \\
\noindent
\\
\textbf{Armando Moda Scarati}
\begin{itemize}
    \item Creazione delle pagine: \texttt{abbonamenti.html}, \texttt{palestra.html} 
    \item Gestione in css delle stesse 
\end{itemize}
\textbf{Alessandro}
\begin{itemize}
    \item Creazione delle pagine: \texttt{home.html}
    \item Gestione in css delle stesse 
\end{itemize}
\textbf{Granziero Luca}
\begin{itemize}
    \item Creazione delle pagine: \texttt{corsi.html} \texttt{error404.html}
    \item Gestione in css delle stesse
    \item Stesura e aggiornamento della relazione 
\end{itemize}
\textbf{Marco}
\begin{itemize}
    \item Creazione delle pagine: \texttt{spa.html} \texttt{contattaci.html}
    \item Gestione in css delle stesse
\end{itemize}

\section{Conclusioni}
Il progetto fornisce una base per un sito web di palestra, pronto per ulteriori sviluppi.

\end{document}
