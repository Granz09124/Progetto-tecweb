\documentclass[a4paper,12pt]{article}
\usepackage[utf8]{inputenc}
\usepackage[italian]{babel}
\usepackage{graphicx}
\usepackage[table]{xcolor}
\usepackage{hyperref}
\usepackage[a4paper, margin=2.5cm]{geometry}
\usepackage[T1]{fontenc}
\usepackage{lmodern}
\usepackage{textcomp}
\usepackage{dirtree}
\usepackage{listings}
\begin{document}

\begin{titlepage}
	\centering
	\vspace*{1cm}
	\includegraphics[width=0.5\textwidth]{Logo_Università_Padova.svg.png} 
	
	\vspace{1.5cm}
	{\Huge \textbf{Tempio di Apollo - Sito Web} \par}
	\vspace{1cm}
	
	{\Large Armando Moda Scarati - 2082864 \par}
	{\Large Granziero Luca - 2075512 \par}
	{\Large Alessandro Morabito - 2137991 \textit{(referente)} \par} 
	{\Large Marco Beltrame - 2111021 \par}
	
	\vspace{1cm}
	{\large \today \par}
	\vspace{1cm}
	\textbf{Sito Web:} \\
	http://tecweb.studenti.math.unipd.it/amorabit \\
	\vspace{1cm}
	\textbf{Credenziali account del sito:}
	
	\begin{center}
		\begin{tabular}{|c|c|c|}
			\hline
			\rowcolor{lightgray}\textbf{Ruolo} & \textbf{Username} & \textbf{Password} \\
			\hline
			USER  & user  & user  \\
			\hline
			PT & pt & pt \\
			\hline
			ADMIN & admin & admin \\
			\hline
		\end{tabular}
	\end{center}
	
	\vfill
\end{titlepage}

\tableofcontents
\newpage

\section{Introduzione}
Questo documento presenta il progetto per la realizzazione del sito web di una palestra che offre i seguenti servizi: sala pesi, corsi fitness e piani personalizzati da un personal trainer. Il sito prevede pagine che offrono una panoramica introduttiva chiara sui servizi che offre, e un'area personale per gli iscritti al sito, in base al loro ruolo all'internio della palestra.

All'interno dei vari file abbiamo utilizzato sempre link relativi, salvo alcune occasioni in cui ci è tornato comodo scrivere link assoluti in maniera dinamica (tramite php, utilizzando \texttt{\textunderscore\textunderscore DIR\textunderscore\textunderscore}) e il file \texttt{.htaccess}, che invece richiede l'utilizzo di percorsi assoluti.

La relazione che ci si aspetta tra le cartelle in termini di profondità nell'albero è la seguente:
\dirtree{%
	.1 /(amorabit/).
	.2 public\_html/.
	.2 internal/.
}


\section{Analisi del target}
Il sito web è destinato:
\begin{itemize}
	\item ai potenziali clienti che, non ancora registrati, possono informarsi sui servizi che la palestra offre.
	\item ai clienti che tramite un'area personale, possono visualizzare le proprie informazioni. 
	\item agli istruttori, la cui area personale è come quella dei clienti, ma possiedono un profilo pubblico
	\item ai personal trainer, tutti anche istruttori, che possono caricare per i propri clienti una scheda tecnica in formato pdf
	\item agli amministratori, che possono visualizzare la lista degli utenti iscritti e modificare le loro informazioni
\end{itemize}

\subsection{SEO}
Il sito vuole rispondere alle seguenti ricerche:
\begin{itemize}
	\item il nome del sito 
	\item tutte le varie ricerche mirate all'argomento palestra, come: salapesi, corsi, personal trainer, pt, attrezzi e bodybuilding
	\item ricerche più generali su argomenti come: allenamento, salute o motivazione.
	
\end{itemize}
\noindent
Le parole chiave che sono state selezionate per rivolgersi sia ad utenti che hanno già frequentato una palestra e che quindi sanno già cosa stanno cercando, sia ai nuovi utenti che non hanno una grande idea di cosa stiano effettivamente cercando.

Per migliorare il ranking del sito sono state svolte le seguenti operazioni:
\begin{itemize}
	\item Ogni pagina presenta un tag <title> univoco che segue una logica descrittiva (es. "Home - Tempio di Apollo").
	Sono stati inseriti meta tag description dettagliati per migliorare il CTR (Click-Through Rate) nella SERP, descrivendo l'offerta della palestra (es. "una palestra con sala pesi e corsi di gruppo, nonché percorsi su misura"), e meta tag keywords per definire il campo semantico della pagina.
	
	\item Utilizzo degli \texttt{<strong>} in base alle parole chiave della pagina
	
	\item La separazione tra struttura, presentazione e comportamento rende più leggero il file visualizzato dai motori di ricerca, ad elimina informazioni non utili nell'\texttt{<head>}, primo contenuto che i motori di ricerca utilizzano per determinare la tipologia di sito
		
	\item È stata fatta attenzione all'accessibilità del sito, di conseguenza migliorandone la qualità e quindi il ranking
	
	\item Il codice HTML è sintatticamente corretto, e sfrutta la semantica HTML in modo appropriato
	
	\item Abbiamo cercato di rendere le pagine leggere mettendo le immagini con dimensioni più vicine possibile a quanto effettivamente utilizzate, favorendo i file \texttt{JPEG} e utilizzando immagini di grandezza diversa quando la grandezza attesa è diversa (ad es. la foto profilo di un istruttore nella pagina di ricerca degli istruttori, che presenta risultati  multipli, e il foto profilo nel profilo pubblico in dettaglio, ove il risultato è singolo)
\end{itemize}

\section{Tecnologie Utilizzate}
\begin{itemize}
	\item \textbf{HTML5}: Utilizzato per definire la struttura semantica delle pagine web, fornendo elementi come header, nav, section e footer per una migliore accessibilità e SEO.
	\item \textbf{CSS}: Impiegato per lo styling e il layout delle pagine, garantendo un design responsive e adattabile a diversi dispositivi.
	\item \textbf{JavaScript}: Integrato per aggiungere interattività e animazioni dinamiche, migliorando l'esperienza utente senza ricaricare la pagina.
	\item \textbf{PHP}: Utilizzato lato backend per la gestione di sessioni utente, l'interazione con il database e aggirare problematiche riscontrate in fase di upload del sito sul server.
	\item \textbf{SQL}: Linguaggio utilizzato per interrogare e gestire database relazionali, essenziale per archiviare e recuperare dati utente, abbonamenti e altre informazioni del sito.
	\item \textbf{LaTeX}: Adottato per la stesura di questa relazione, permettendo la creazione di documenti di alta qualità con formattazione professionale.
	\item \textbf{LLM}: Utilizzati per la generazione di contenuto generico e talvolta di immagini, per evitare di far riferimento a persone specifiche e di avere problemi con i copyright. La maggior parte delle immagini generate tramite LLM presentano la classica stella gemini nell'angolo inferiore destro. Sono un'eccezione il logo del sito, le immagini presenti nelle pagine di errore, e qualche personal trainer, che non hanno questo segno identificati. 
	
	Per quanto riguarda il contenuto non di immagine, i testi della pagina home e i dati presenti nel database sono stati prima generati, successivamente perfezionati manualmente.
\end{itemize}

\section{Utenti presenti nel sito}
All'interno del sito troviamo i seguenti tipi di utenti:
\begin{itemize}
	\item \textbf{cliente}
	\item \textbf{personal trainer}
	\item \textbf{amministratore}
\end{itemize}
Anche se abbiamo distinto i personal trainer dagli istruttori, al lato pratico l'area personale degli istruttori risulta identica a quella di un cliente semplice: è solo l'\textbf{istruttore} che, come si vedrà, ha delle funzionalità aggiuntive. La differenza tra istruttori e clienti non sta nell'area personale, bensì nel fatto che gli istruttori possiedono un \textbf{profilo pubblico} raggiungibile tramite la pagina \texttt{cerca-istruttore.php}.

\subsection{Cliente}
\begin{itemize}
	\item Visualizzazione dell'abbonamento attuale, se attivo, oppure del fatto che l'abbonamento è scaduto
	\item Se presente, visualizzazione della scheda caricata dal personal trainer (pdf)
	\item Gestione del proprio account (modifica indirizzo e-mail, password, telefono).
\end{itemize}

\subsection{Personal Trainer}
\begin{itemize}
	\item Profilo pubblico, visualizzabile da chiunque.
	\item Visualizzazione della lista dei clienti attivi
	\item Possibilità di caricare file pdf per i clienti (schede tecniche)
	\item Lista dei clienti che ha seguito e sta seguendo
	\item Gestione del proprio account (modifica indirizzo e-mail, password, telefono).
\end{itemize}

\subsection{Amministratore}
\begin{itemize}
	\item Visualizzazione lista utenti, e modifica delle informazioni personali
	\item Visualizzazione lista clienti, e modifica delle informazioni personali
	\item Visualizzazione lista personal trainer, e modifica delle informazioni personali. Tra queste, \textit{specializzazione} e \textit{qualifica} vengono scritte a schermo sulla pagina del profilo del personal trainer
	\item Modifica delle credenziali
\end{itemize}


\section{Struttura del sito}
\subsection{Procedimento generale}
Alla fine del capitolo è presente lo schema della struttura del sito, visualizzata in base ai link che può seguire l'utente. I breadcrumb sono sempre relativi al primo livello (quello della homepage), ad eccezione delle 3 possibili aree personali, una per tipologia di utenza. L'utente visualizza sempre lo stesso indirizzo sulla barra del browser, ma la schermata varia in base al suo ruolo.

Il motivo per cui abbiamo adottato questa struttura (con collegamenti circolari) è che volevamo permettere all'utente di raggiungere pagine utili direttamente dalla sua area personale.

Si fa notare che "Logout" non è effettivamente una pagina, ma semplicemente un link che fa effettuare l'operazione di logout e reindirizza l'utente, ora non autenticato, alla homepage del sito. Inoltre, "Area Personale" e "Logout" prendono il posto di "Login" e "Registrati" nel menu principale, una volta effettuato il login.

Si fa presente, inoltre, che non sono presenti le pagine php pubbliche che permettono di effettuare le varie operazioni di cui si è parlato nel capitolo precedente. Le pagine invece che abbiamo ritenuto non dover essere accessibili (come quelle delle aree personali della singola tipologia di utente) sono state inserite all'interno della cartella \texttt{internal/}, non visualizzabile pubblicamente in quanto esterna a \texttt{public\textunderscore html/}.


\begin{center}
	\includegraphics[width=.8\linewidth]{struttura_sito.png}
\end{center}

\subsection{Convenzioni interne} \label{conv-interne}
Vengono riportate le convenzioni interne del sito:
\begin{itemize}
	\item I link, sia quelli visitati e non visitati, vengono presentati come sottolineati, in particolare i link non visitati hanno un colore di testo chiaro vicino all'azzurro, mentre i link visitati hanno un colore di testo viola, inoltre sono entrambi posti su sfondo nero. Questi colori sono stati scelti in quanto fanno parte della palette di colori utilizzata. 
	\item Nelle tabelle, i colori delle righe dispari e pari sono sempre diversi tra loro, e sempre gli stessi all'interno del sito
	\item Tutte le voci del menu e del breadcrumb sono raggiungibili tramite \texttt{tab}, anche le voci che non sono link in quanto corrispondono alla pagina corrente. In questo modo, per passare dalla voce "Home" ad "Abbonamenti" bisogna premere il tasto \texttt{tab} due volte, anche se la pagina attuale è "Palestra", la cui voce si trova esattamente in mezzo tra le due.
	\item In fase di stampa, il colore principale della palestra, il giallo, viene trasformato in nero per favorire la lettura. Scompare il menu, e si preserva il breadcrumb. All'interno del gruppo si è deciso individualmente quali immagini mantenere e quali stampare, immaginando l'utilizzo dell'utente.
\end{itemize}

\subsection{Struttura del database}
A seguire la struttura del database. Come scritto sopra, la tabella \texttt{Messaggio\_Contattaci} è presente unicamente per dare un comportamento al pulsante di invio messaggi. Le informazioni vengono salvate sul database, ma non vengono mai riflesse a schermo.
\begin{center}
	\includegraphics[width=.9\linewidth]{diagramma_er.png}
\end{center}


\section{Progettazione}
La progettazione è iniziata poco dopo l'inizio dell'assegnazione dei gruppi.
\noindent
\\
Si è rapidamente deciso che l'argomento del sito sarebbe stata una palestra. Dopo aver deciso il nome, abbiamo iniziato a stilare una lista di operazioni che avremmo potuto permettere agli utenti. Successivamente, ci siamo focalizzati sul contenuto e lo stile del sito, nonché sulla struttura generale. È stato in quest'occasione che abbiamo deciso di introdurre dei cicli nella struttura gerarchica e le breadcrumb.
\noindent
\\
Abbiamo riscontrato problemi quando abbiamo dovuto aggiungere del comportamento al sito, in particolare nella parte di backend: nell'analisi dei requisiti non abbiamo incluso le funzionalità che avrebbero dovuto effettivamente essere presenti in quanto progetto, e questo ci si è ritorto contro verso la fine del progetto, quando abbiamo dovuto aggiungere ulteriori funzionalità.
\noindent
\\
Ad ogni incontro sincrono abbiamo messo per iscritto quello che era stato fatto, quello che avremmo dovuto fare tra quell'incontro e quello successivo, e quali sarebbero stati gli argomenti da affrontare in seguito. Questo ci ha permesso di tenere un buon ritmo ed effettuare modifiche rapidamente quando ci siamo accorti di essere finiti fuori scope.


\section{Presentazione}
\subsection{CSS}
All'interno del nostro CSS vengono utilizzate delle variabili, che ci permettono controllare facilmente i vari contrasti e i colori utilizzati.
Vengono utilizzati tre diversi fogli di stile, ognuno con uno scopo diverso: 
\begin{itemize}
	\item \texttt{style.css}: per lo stile da desktop 
	\item \texttt{mini.css}: per lo stile da mobile 
	\item \texttt{print.css}: per lo stile da stampante
\end{itemize}
\noindent
È stata presa questa decisione è stata presa per facilitare la gestione dei css e diminuire i tempi di risposta del sito.

\subsection{Immagini e icone}
La maggior parte delle immagini vengono archiviate prevalentemente in formato JPG, con una dimensione massima che non supera il MB, ad eccezione di questo troviamo le immagini generate per le pagine error e le icone, che vengono salvate in formato WEBP, e l'immagine del logo che è salvata in formato PNG ma non supera la grandezza di 1MB. Questa scelta è stata adottata per ridurre i tempi di caricamento delle pagine; tuttavia, l’utilizzo di questo formato potrebbe comportare possibili problemi di retrocompatibilità, dovuti alla sua introduzione relativamente recente.


\section{Accessibilità}

\subsection{Palette dei colori e contrasti}
La palette dei colori è stata scelta per garantire un elevato contrasto tra testi e sfondi, rispettando le linee guida WCAG. Ad esempio, testi chiari su sfondi scuri per una migliore leggibilità, evitando combinazioni che potrebbero causare difficoltà a utenti con disabilità visive.

\bigskip
\noindent
\textbf{Palette dei colori utilizzata:}

\begin{center}
	\begin{tabular}{|c|c|c|}
		\hline
		\rowcolor{lightgray} \textbf{Colore} & \textbf{Codice HEX} & \textbf{Utilizzato per} \\
		\hline
		\cellcolor[HTML]{c0ebfb}\hspace{2cm} & \texttt{\#c0ebfb}  & Links non visitati \\
		\hline
		\cellcolor[HTML]{965cfd}\hspace{2cm} & \texttt{\#965cfd}  & Links non visitati \\
		\hline
		\cellcolor[HTML]{ffee55}\hspace{2cm} & \texttt{\#ffee55}  & Colore principale chiaro \\
		\hline
		\cellcolor[HTML]{ffcc00}\hspace{2cm} & \texttt{\#ffcc00}  & Colore principale \\
		\hline
		\cellcolor[HTML]{ee8800}\hspace{2cm} & \texttt{\#ee8800}  & Colore principale scuro\\
		\hline
		\cellcolor[HTML]{333322}\hspace{2cm} & \texttt{\#333322} &  Sfondo per tabelle \\
		\hline 
		\cellcolor[HTML]{181211}\hspace{2cm} & \texttt{\#181211}  & Sfondo \\
		\hline
		\cellcolor[HTML]{111111}\hspace{2cm} & \texttt{\#111111}  & Sfondo header\\
		\hline
		\cellcolor[HTML]{dddddd}\hspace{2cm} & \texttt{\#dddddd}  & Testo generico \\
		\hline
	\end{tabular}
\end{center}
I colori principali sono stati utilizzati solamente in breadcrumb e heading. In quest'ultimo caso, come spiegato successivamente nella sezione \ref{sec-acc} dedicata ai test di accessibilità, è stato utilizzato un gradiente sul testo, che fa restituire a certi tool risultati incoerenti con quanto visualizzato effettivamente.
\bigskip 
Eventuali altri colori, impiegati in modo occasionale, sono stati comunque scelti e verificati nel rispetto dei principi di accessibilità.

\subsection{Lingue straniere}
Il sito supporta contenuti in italiano come lingua principale, ma utilizza attributi \texttt{lang} per le parole straniere, permettendo una migliore accessibilità tramite screen reader.

In merito a questo, nella pagina \texttt{Home} è presente un \texttt{lang="fr"} per \textit{tapis roulant} che, non capiamo il perché, NVDA non legge in francese, nonostante notifichi il cambio di lingua.

\subsection{Screen reader}
La separazione tra struttura, presentazione e comportamento garantisce che il sito sia accessibile a tutti gli utenti, inclusi quelli che utilizzano screen reader o dispositivi assistivi. L'HTML semantico facilita la navigazione e la comprensione del contenuto.
L’accessibilità per gli utenti di screen reader è stata migliorata attraverso l’uso appropriato della semantica HTML5 e degli attributi ARIA e alt per le immagini, dove necessario. Questo è stato reso possibile attraverso l'uso di screen reader come NVDA che ci hanno aiutato a simulare l'esperienza di navigazione di un utente che ne necessita il bisogno.

\paragraph{Aiuti alla navigazione}
All’inizio di ogni pagina è presente un collegamento di salto al contenuto principale (“Vai al contenuto”), posizionato come primo elemento del DOM. Questo link è inizialmente nascosto e diventa visibile quando riceve il focus da tastiera, consentendo agli utenti che navigano senza mouse di bypassare elementi ripetitivi come l’intestazione e il menu di navigazione e raggiungere direttamente il contenuto principale racchiuso nel tag \texttt{<main>}.

\paragraph{Breadcrumb}
Come scritto nella sezione \ref{conv-interne}, si può utilizzare \texttt{tab} all'interno delle breadcrumb per raggiungere anche il livello che comunica la pagina attuale. La decisione è stata dovuta anche al fatto che abbiamo deciso di esprimere i breadcrumb come lista ordinata, e ci sembrava controintuitivo che il tasto \texttt{tab} saltasse sempre l'ultimo livello.

\paragraph{Tabelle dati}
Le tabelle utilizzate per visualizzare gli orari dei corsi sono state progettate per essere facilmente interpretabili dagli screen reader. In particolare:
\begin{itemize}
	\item è stato utilizzato un paragrafo descrittivo nascosto visivamente, collegato alla tabella tramite l’attributo \texttt{aria-describedby}. Il tag \texttt{<caption>} è stato omesso in quanto tanto vicino all'heading \texttt{<h2>}, e la tabella è l'unico contenuto sotto quest'heading. Questo consente allo screen reader di leggere una descrizione introduttiva del contenuto della tabella prima della navigazione delle celle. Abbiamo notato che NVDA legge due volte la descrizione della tabella, una volta che raggiunge la tabella: utilizzare \texttt{aria-hidden} sulla descrizione risolve il problema, ma abbiamo visto può provocare problemi di compatibilità, quindi non è stato inserito;
	\item le celle di intestazione (\texttt{<th>}) utilizzano l’attributo \texttt{scope} per associare correttamente le celle di dati alle rispettive intestazioni di riga e di colonna, permettendo una lettura contestualizzata delle informazioni.
\end{itemize}

\paragraph{Menu degli abbonamenti}
Nella sezione dedicata agli abbonamenti, la selezione delle diverse categorie è implementata tramite un menu di navigazione semantico contenente pulsanti standard (\texttt{<button>}). L’utilizzo di elementi HTML nativi garantisce la corretta accessibilità da tastiera e una gestione automatica del focus, evitando l’uso di ruoli ARIA complessi e mantenendo il codice semplice e semanticamente corretto.

\paragraph{Gestione della lingua}
Per assicurare una corretta pronuncia da parte dei sintetizzatori vocali, i termini in lingua inglese presenti all’interno del testo italiano sono racchiusi in elementi \texttt{<span>} con attributo \texttt{lang="en"}.


\subsection{Compatibilità}
I contenuti di ogni pagina sono stati disposti in modo tale da essere facilmente accessibili indipendentemente dal dispositivo utilizzato.
Abbiamo adottato un menù ad hamburger con css puro per gli utenti da mobile, facilitandone la navigazione; le varie tabelle presenti inoltre sono state linearizzate in modo da facilitarne la lettura su diversi dispositivi.

\section{Test effettuati}

\subsection{Accessibilità} \label{sec-acc}
Tutti i test sono stati effettuati sui seguenti browser:
\begin{itemize}
	\item firefox
	\item chrome
	\item brave
	\item microsoft edge
\end{itemize}
Per quanto riguarda l'accessibilità, abbiamo effettuato:
\begin{enumerate}
	\item test manuali per controllare la responsiveness del sito, tramite \texttt{Ispeziona elemento}
	\item test manuali per controllare la navigabilità del sito tramite \texttt{<tab>}
	\item test manuali per controllare la qualità di lettura del sito tramite \textbf{NVDA}, sia in modalità desktop sia in modalità \textit{mobile}
	\item test automatici utilizzando l'estensione \textbf{WCAG Contrast Checker}. Abbiamo rilevato contrasti livello AAA, con falsi positivi sugli heading che sembrano avere contrasto unitario. Il motivo è che è stato deciso di utilizzare un gradiente come colore del testo, e per farlo si è dovuto impostare lo sfondo di un certo colore, il testo trasparente, quindi applicare una maschera in modo che lo sfondo venga applicato solamente dove compare il testo. Abbiamo verificato che l'estensione \textbf{Dark Reader}\footnote{Dark reader: \url{https://darkreader.org/}}, se attivata, rende il colore dei titoli con contrasto insufficiente. Dal momento che il sito utilizza un tema scuro, tuttavia, e il fatto che l'estensione provoca questo tipo di problemi in molti siti, abbiamo deciso di focalizzarci su altri ostacoli per l'accessibilità.
	\item test automatici utilizzando l'estensione \textbf{w3b1y}
\end{enumerate}
Abbiamo deciso di non utilizzare, quando possibile, la pagina di errore 403, in quanto non abbiamo ritenuto fosse utile per un utente nel nostro contesto. Abbiamo deciso di effettuare redirect all'area personale oppure alla pagina di login, qualora un utente dovesse cercare di raggiungere una pagina a lui irraggiungibile. La pagina di errore 403 è stata comunque impostata nel file \texttt{.htaccess}, e verrà utilizzata qualora il server chieda di accedere a pagine senza i permessi adatti. Questo succederebbe ad esempio provando ad accedere alla pagina (directory) \texttt{/amorabit}. Anche in questo caso, tuttavia, è stata utilizzata una regola di riscrittura dell'indirizzo, seppur a un differente livello. Provare ad accedere a \texttt{amorabit/} reindirizzerà a \texttt{amorabit/home.php}.

\subsection{Privacy}
Dal momento che abbiamo permesso ai personal trainer di caricare dei file da destinare a uno specifico utente, abbiamo deciso di porre quei file all'interno della cartella \texttt{internal/utente/utente-semplice/uploads/schede/}, non accessibile al pubblico. Gli utenti dopo aver cliccato sul link apposito, o accedendo a \texttt{/download-scheda.php} potranno visualizzare la propria, e solamente la propria, scheda.

\subsection{Sicurezza}
Per quanto riguarda la sicurezza, abbiamo effettuato due tipologie di test:
\begin{enumerate}
	\item test statici, utilizzando il tool "Snyk" come estensione di VSCode. Questo ha riportato alcune vulnerabilità in fase di sviluppo, che abbiamo risolto.
	\item test dinamici, non automatizzati. Per quanto i test automatizzati siano incredibilmente più efficaci ed efficienti, dal momento che ci siamo trovati tardi ad effettuarli, avremmo dovuto farlo sui server dell'università. Questo non ci risulta essere legale, e comunque sarebbe rischioso dal momento che non abbiamo mai utilizzato strumenti di analisi dinamica prima (potenzialmente avremmo potuto congestionare il server oppure trovarci con una connessione limitata che ci avrebbe impedito di continuare lo sviluppo)
\end{enumerate}
Inoltre, a tutti i file php in \texttt{public\textunderscore html/} è stato assegnato un permesso 644, in modo che gli utenti esterni possano visualizzare la pagina (l'ideale sarebbe stato 640, ma non siamo sicuri del funzionamento in un server in condivisione come quello dell'università). I file php dentro \texttt{internal/} invece hanno assegnato permessi 600, in modo che solamente il server possa leggerli. Non siamo riusciti in realtà a effettuare test in merito a questa tipologia di permessi: le pagine pubbliche con permessi 600 risultano comunque raggiungibili tramite browser. Sospettiamo che il motivo sia la connessione ssh che viene utilizzata per la connessione al sito. In ogni caso, la cartella \texttt{internal/} non è presente all'interno di \texttt{public\textunderscore html} e quindi non è accessibile (salvo vulnerabilità) tramite browser. Per ripetibilità, lo shell script (\texttt{secure.sh}) che associa i permessi è il seguente:
\begin{lstlisting}[language=bash]
#!/bin/sh
find public_html -name '*.php' -exec chmod 644 {} \;
find internal -name '*.php' -exec chmod 600 {} \;
\end{lstlisting}
Infine, abbiamo provato a caricare file pdf e non, di grandezze diverse, anche modificando l'estensione.
\subsubsection{Cosa non abbiamo fatto}
Non siamo riusciti a mettere bene in sicurezza le tabelle del database: l'ideale sarebbe stato avere una connessione per database, ma questo non abbiamo potuto farlo in quanto non possiamo creare nuovi utenti tramite phpmyadmin. Un'alternativa che abbiamo considerato è stato limitare l'accesso al database in base all'indirizzo di connessione, permettendo l'accesso a certe tabelle solamente tramite connessione verso localhost. Tuttavia non avremmo potuto effettivamente sperimentare se le connessioni fossero sicure, potendoci collegare solamente tramite localhost. Infatti, non avremmo potuto confermare neanche il funzionamento del sito in caso di connessione da host remoto. Inoltre, il sito in questo modo non sarebbe più diventato trasportabile, come richiesto dalla specifica della consegna. Per questi motivi non ci siamo soffermati a pensare a come effettivamente avremmo potuto sfruttare questa tipologia di permessi.


\section{Divisione dei compiti}
Vengono qui esposti i vari compiti svolti da ciascun membro del team di progetto: \\
\noindent
\\
\textbf{Armando Moda Scarati}
\begin{itemize}
    \item Gestione strutturale e logica delle pagine principali:
    \begin{itemize}
        \item \texttt{area-personale.php}: implementazione del sistema di \textit{routing} che indirizza l'utente alla dashboard corretta (admin, pt, cliente) in base al ruolo in sessione.
        \item \texttt{palestra.php} e \texttt{abbonamenti.php}: assemblaggio modulare dei template HTML e gestione della logica di visualizzazione.
    \end{itemize}
    \item Sviluppo Backend (PHP) e gestione dati:
    \begin{itemize}
        \item Creazione delle pagine \texttt{utente-admin.php}, \texttt{utente-pt.php} e \texttt{utente-semplice.php} in collaborazione con Luca Granziero.
        \item Creazione delle pagine \texttt{lista-clienti.php}, \texttt{lista-pt.php}, \texttt{lista-utenti.php} e \texttt{lista-clienti-pt.php} in collaborazione con Luca Granziero.
        \item Implementazione delle logiche \textbf{CRUD} (Create, Read, Update, Delete) per la gestione degli utenti e delle assegnazioni.
        \item Gestione delle risposte \textbf{JSON} per le richieste asincrone di modifica dati, permettendo il popolamento dinamico delle modali.
    \end{itemize}
    \item Sviluppo JavaScript:
    \begin{itemize}
        \item \texttt{dashboard.js}: gestione della navigazione interna (sidebar laterale) e dei filtri dinamici per le liste amministrative.
        \item \texttt{selettoreAbbonamenti.js}: manipolazione del DOM per la gestione accessibile dei \textit{tabs} (schede), controllo delle classi \textit{active} e degli attributi ARIA.
    \end{itemize}
    \item Gestione CSS (desktop, mobile e print) per le rispettive pagine.
    \item Conclusione e revisione della relazione.
\end{itemize}
\textbf{Alessandro Morabito}
\begin{itemize}
	\item Creazione e gestione dei file:
	\begin{itemize}
		\item \texttt{cerca-istruttore.php} (con css mobile, desktop e stampa)
		\item \texttt{profilo-istruttore.php} (con css mobile, desktop e stampa)
		\item \texttt{home.php} (con css mobile, desktop e stampa)
		\item \texttt{download-scheda.php}
		\item \texttt{upload-messaggio.php} (utilizzato solamente per dare un significato al form contattaci. I dati vengono in realtà salvati nel database)
		\item \texttt{lista-clienti-pt.php} i controlli sul file pdf (scheda da parte del personal trainer) caricato (grandezza posta a 1MiB in quanto possibilmente vincolato da configurazione php, controllo estensione file, controllo MIME file)
		\item \texttt{utente-semplice.php} pulsante \texttt{Visualizza PDF (...)} per la visualizzazione della scheda caricata da un personal trainer
		\item \texttt{caricaScheda.js}
		\item \texttt{caricaMessaggio.js}
		\item \texttt{cercaIstruttore.js}
		\item menù mobile (css puro)
		\item \texttt{.htaccess} (impostati riferimenti a errore 403 e 404, riscrittura condizionata URL: se si prova ad accedere alla \textit{root} del sito, viene aggiunto \texttt{/home.php} all'indirizzo)
	\end{itemize}
	\item Gestione in css delle stesse 
	\item conclusione e revisione della relazione
\end{itemize}
\textbf{Luca Granziero}
\begin{itemize}
	\item Creazione e gestione dei file:
        \begin{itemize}
            \item \texttt{corsi.php} (con css mobile, desktop e stampa)
            \item \texttt{salapesi.php} (con css mobile, desktop e stampa)
            \item \texttt{403.php} (con css mobile, desktop e stampa)
            \item \texttt{404.php} (con css mobile, desktop e stampa)
               \item Creazione delle pagine \texttt{utente-admin.php}, \texttt{utente-pt.php} e \texttt{utente-semplice.php} in collaborazione con Armando Moda Scarati
        \item Creazione delle pagine \texttt{lista-clienti.php}, \texttt{lista-pt.php}, \texttt{lista-utenti.php} e \texttt{lista-clienti-pt.php} in collaborazione con Armando Moda Scarati.
        \end{itemize}
    \item Gestione in css delle stesse 
	\item Stesura e aggiornamento della relazione 
\end{itemize}
\textbf{Beltrame Marco}
\begin{itemize}
    \item Gestione della pagine: \texttt{abbonamenti.html}, \texttt{contattaci.html},  \texttt{login.html},  \texttt{registrati.html}
    \item Creazione della prima versione del database \texttt{db.sql} e mantenimento dello stesso
	\item Creazione e gestione dei file:
	\begin{itemize}
		\item \texttt{login.php} (con css mobile, desktop e stampa)
		\item \texttt{registrati.php} (con css mobile, desktop e stampa)
		\item \texttt{contattaci.php} (con css mobile, desktop e stampa)
        \item \texttt{processa-registrazione.php} (gestisce la validazione lato server dei dati utente, verifica l'univocità di email e codice fiscale e finalizza l'iscrizione inserendo i dati nel database in modo sicuro)
        \item \texttt{processa-login.php} (autentica l'utente verificando le credenziali e assegna il ruolo specifico tra cui: cliente, admin o pt)
        \item \texttt{header.php} (inietta il template del menu di navigazione appropriato differenziando tra utente loggato o ospite, gestendo i relativi attributi tra cui aria-current e tabindex e lang)
        \item \texttt{processa-registrazione.php} (gestisce la validazione lato server dei dati utente, verifica l'univocità di email e codice fiscale e finalizza l'iscrizione inserendo i dati nel database in modo sicuro)
		\item \texttt{caricaLogin.js} (usato per mostrare eventuali errori sull'html)
		\item \texttt{caricaRegistrazione.js} (usato per mostrare eventuali errori sull'html)
		\item \texttt{mostraPassword.js}
	\end{itemize}
	\item Gestione in css delle stesse 
	\item Conclusione e revisione della relazione
    
\end{itemize}


\section{Conclusioni e sviluppi futuri}
Il progetto costituisce una base strutturata per lo sviluppo di un sito web per una palestra, progettato in modo da consentire agevolmente future espansioni e l’implementazione di ulteriori funzionalità, quali ad esempio:
\begin{itemize}
	\item Gestione dell'abbonamento, visualizzazione degli orari, posti disponibili e istruttori dei corsi a cui l'utente è iscritto (per il cliente)
	\item Calendario con le disponibilità e le fasce orarie già occupate dagli altri clienti (per il personal trainer)
	\item Statistiche sull'utilizzo della palestra, corsi più frequentati e tassi di retention (amministratore)
\end{itemize}
\end{document}
