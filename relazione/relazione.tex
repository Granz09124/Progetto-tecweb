\documentclass[a4paper,12pt]{article}
\usepackage[utf8]{inputenc}
\usepackage[italian]{babel}
\usepackage{graphicx}
\usepackage[table]{xcolor}
\usepackage{hyperref}
\usepackage[a4paper, margin=2.5cm]{geometry}
\usepackage[T1]{fontenc}
\usepackage{lmodern}
\usepackage{textcomp}
\usepackage{dirtree}
\begin{document}

\begin{titlepage}
	\centering
	\vspace*{1cm}
	%   \includegraphics[width=0.5\textwidth]{Logo_Università_Padova.svg.png} 
	
	\vspace{1.5cm}
	{\Huge \textbf{Tempio di Apollo - Sito Web} \par}
	\vspace{1cm}
	
	{\Large Armando Moda Scarati - 2082864 \par}
	{\Large Granziero Luca - 2075512 \par}
	{\Large Alessandro Morabito - 2137991 \textit{(referente)} \par} 
	{\Large Marco Beltrame - 2111021 \par}
	
	\vspace{1cm}
	{\large \today \par}
	\vspace{1cm}
	\textbf{Sito Web:} \\
	tecweb.studenti.math.unipd.it/amorabit \\
	\vspace{1cm}
	\textbf{Credenziali account del sito:}
	
	\begin{center}
		\begin{tabular}{|c|c|c|}
			\hline
			\rowcolor{lightgray}\textbf{Ruolo} & \textbf{Username} & \textbf{Password} \\
			\hline
			USER  & user  & user  \\
			\hline
			PT & pt & pt \\
			\hline
			ADMIN & admin & admin \\
			\hline
		\end{tabular}
	\end{center}
	
	\vfill
\end{titlepage}

\tableofcontents
\newpage

\section{Introduzione}
Questo documento presenta il progetto per la realizzazione del sito web di una palestra che offre i seguenti servizi: sala pesi, corsi fitness e piani personalizzati da un personal trainer. Il sito prevede pagine che offrono una panoramica introduttiva chiara sui servizi che offre.
Prevede inoltre le seguenti tipologie di area personale: per i clienti, per gli istruttori, personal trainer, e per gli admin.

All'interno dei vari file abbiamo utilizzato sempre link relativi, salvo alcune occasioni in cui ci è tornato comodo scrivere link assoluti in maniera dinamica (tramite php) e il file \texttt{.htaccess}, che invece richiede di utilizzare percorsi assoluti.

La relazione, in termini di profondità nell'albero, che ci si aspetta tra le cartelle è la seguente:
\dirtree{%
	.1 /(amorabit/).
	.2 public\_html/.
	.2 internal/.
}


\section{Analisi del target}
Il sito web è destinato:
\begin{itemize}
	\item ai potenziali clienti che, non ancora registrati, possono informarsi sui servizi che la palestra offre.
	\item ai clienti che tramite un area personale, possono visualizzare le proprie informazioni. 
	\item agli istruttori, la cui area personale è come quella dei clienti, ma possiedono un profilo pubblico
	\item ai personal trainer, tutti anche istruttori, che possono caricare per i propri clienti una scheda tecnica in formato pdf, di grandezza minore di 1MiB
	\item agli amministratori, che possono visualizzare la lista degli utenti iscritti e modificare le loro informazioni
\end{itemize}

\subsection{SEO}
Il sito vuole rispondere alle seguenti ricerche:
\begin{itemize}
	\item il nome del sito 
	\item tutte le varie ricerche mirate all'argomento palestra, come: salapesi, corsi, personal trainer, pt, attrezzi e bodybuilding
	\item ricerche più generali su argomenti come: allenamento, salute o motivaizione.
	
\end{itemize}
\noindent
Le parole chiave che sono state selezionate per rivolgersi sia ad utenti che hanno già frequentato una palestra e che quindi sanno già cosa stanno cercando, sia ai nuovi utenti che non hanno una grande idea di cosa stiano effettivamente cercando.
\noindent
\\
\\
Per migliorare il ranking del sito sono state svolte le seguenti operazioni:
\begin{itemize}
	\item Ogni pagina presenta un tag <title> univoco che segue una logica descrittiva (es. "Home - Tempio di Apollo" o "Palestra - Offre").
	Sono stati inseriti meta tag description dettagliati per migliorare il CTR (Click-Through Rate) nella SERP, descrivendo l'offerta della palestra (es. "una palestra con sala pesi e corsi di gruppo, nonché percorsi su misura"), e meta tag keywords per definire il campo semantico della pagina.
	
	\item La separazione tra struttura (HTML), presentazione (CSS) e contenuto garantisce che il sito sia accessibile a tutti gli utenti, inclusi quelli che utilizzano screen reader o dispositivi assistive. L'HTML semantico facilita la navigazione e la comprensione del contenuto.
	\item Poiché l’accessibilità rappresenta un fattore indiretto di posizionamento, sono state implementate diverse soluzioni tecniche volte a migliorare sia l’esperienza degli utenti che utilizzano screen reader, sia l’indicizzazione da parte dei motori di ricerca. In particolare:
	
	Gestione della lingua: affrontata in modo approfondito nella sezione dedicata all’accessibilità.
	
	Attributi alt: tutte le immagini con valore informativo sono corredate da testi alternativi descrittivi (ad esempio: “Foto di una statua greca che lancia un disco…”), consentendo la comprensione del contenuto visivo e favorendone l’indicizzazione. (SPOSTARE MAGARI IN SEZIONE SCREEN READER E DIRE SOLAMENTE CHE VIENE AFFRONTATA ANCHE QUI DOPO COME PER LA GESTIONE DELLA LINGUA?)
	
	\item Ogni pagina include, attraverso il metatag keywords, le parole chiave che ne descrivono il contenuto.
\end{itemize}

\section{Tecnologie Utilizzate}
\begin{itemize}
	\item \textbf{HTML5}: Utilizzato per definire la struttura semantica delle pagine web, fornendo elementi come header, nav, section e footer per una migliore accessibilità e SEO.
	\item \textbf{CSS}: Impiegato per lo styling e il layout delle pagine, garantendo un design responsive e adattabile a diversi dispositivi.
	\item \textbf{JavaScript}: Integrato per aggiungere interattività e animazioni dinamiche, migliorando l'esperienza utente senza ricaricare la pagina.
	\item \textbf{PHP}: Utilizzato per generare contenuti dinamici lato server, come la gestione di sessioni utente e l'interazione con database per funzionalità avanzate.
	\item \textbf{SQL}: Linguaggio utilizzato per interrogare e gestire database relazionali, essenziale per archiviare e recuperare dati utente, abbonamenti e altre informazioni del sito.
	\item \textbf{LaTeX}: Adottato per la stesura di questa relazione, permettendo la creazione di documenti di alta qualità con formattazione professionale.
	\item \textbf{LLM}: Utilizzati per la generazione di immagini come: il logo, le immagini di sfondo presenti nelle pagine di error403 e error404 e le immagini dei vari PT presenti. Usato anche per la popolazione di alcune descrizioni dei PT, delle recensioni presenti della pagina home e del database. (CONTROLLATE SE VA BENE COSI!!!!!!!)
\end{itemize}

\section{Utenti presenti nel sito}
All'interno del sito troviamo tre tipi di utenti:\\
\texttt{utente/user} \\
\texttt{istruttori e personal trainer} \\
\texttt{admin/amministratore}. \\
\noindent
\\

Di seguito sono riportate le funzionalità disponibili per i ruoli User e Admin:
\subsection{USER}
\begin{itemize}
	\item Gestione semplice della disdetta dell’abbonamento (valutare se mantenere).
	\item Visualizzazione della data di attivazione, scadenza e dello stato attuale dell’abbonamento (attivo, in scadenza, scaduto).
	\item Visualizzazione orari, posti disponibili e istruttori dei corsi a cui    l’utente è iscritto.
	\item Calendario dei corsi a cui si è iscritti e degli appuntamenti con il personal trainer.
	\item Visualizzazione della scheda di allenamento personalizzata (se presente personal trainer).
	\item Visualizzazione del piano alimentare personalizzato (se presente personal trainer).
	\item Possibilità di lasciare recensioni (da valutare per questioni di sicurezza).
	\item Gestione del proprio account (cancellazione account, modifica dei dati).
\end{itemize}

\subsection{PT}
\begin{itemize}
	\item Profilo pubblico, visualizzabile da chiunque.
	\item Calendario con le disponibilità e le fasce orarie già occupate da altri utenti.
	\item Possibilità di caricare scheda di allenamento e piano alimentare per i clienti che lo richiedono.
	\item Lista dei clienti (confermata), con possibilità di chat con ogni cliente non in tempo reale (da valutare).
\end{itemize}

\subsection{ADMIN}
\begin{itemize}
	\item Possibilità di creare account personal trainer o convertire un account utente in account PT.
	\item Visualizzazione e gestione di tutti gli abbonamenti attivi, sospesi o scaduti.
	\item Gestione del profilo utente (schede, corsi, prenotazioni, PT assegnati, scadenze, ecc.).
	\item Gestione degli istruttori/personal trainer: calendario, disponibilità, schede clienti, retribuzione (possibile).
	\item Dashboard con panoramica di ricavi, abbonamenti attivi, nuovi iscritti e lista degli iscritti attivi.
	\item Statistiche sull’utilizzo della palestra, corsi più frequentati e tassi di retention (possibile).
	\item Cancellazione account (sia utente che PT) tramite dashboard.
	\item Possibilità di gestione dei corsi (modifica della capienza e degli orari).
\end{itemize}


\section{Struttura del sito}
\subsection{Procedimento generale}
Alla fine del capitolo è presente lo schema della struttura del sito, visualizzata in base ai link che può seguire l'utente. I breadcrumb sono sempre relativi al primo livello (quello della homepage), ad eccezione delle 3 possibili aree personali, una per tipologia di utenza. L'utente visualizza sempre lo stesso indirizzo sulla barra del browser, ma la schermata varia in base al suo ruolo.

Il motivo per cui abbiamo deciso questa struttura (con collegamenti circolari) è che volevamo permettere all'utente di raggiungere pagine utili direttamente dalla sua area personale.

Si fa notare che "Logout" non è effettivamente una pagina, ma semplicemente un link che fa effettuare l'operazione di logout e reindirizza l'utente, ora non autenticato, alla homepage del sito. Inoltre, "Area Personale" e "Logout" prendono il posto di "Login" e "Registrati" nel menu principale, una volta effettuato il login.

Si fa presente, inoltre, che non sono presenti le pagine php pubbliche che permettono di effettuare le varie operazioni di cui si è parlato nel capitolo precedente. Le pagine invece che abbiamo ritenuto non dover essere accessibili (come quelle delle aree personali della singola tipologia di utente) sono state inserite all'interno della cartella \texttt{internal/}, non visualizzabile pubblicamente in quanto esterna a \texttt{public\textunderscore html/}.


\begin{center}
	\includegraphics[width=.8\linewidth]{struttura_sito.png}
\end{center}

\subsection{Convenzioni interne}
Vengono riportate le convenzioni interne del sito:
\begin{itemize}
	\item I link, sia quelli visitati e non visitati, vengono presentati come sottolineati, in particolare i link non visitati hanno un colore di testo chiaro vicino all'azzurro, mentre i link visitati hanno un colore di testo viola, inoltre sono entrambi posti su sfondo nero. Questi colori sono stati scelti in quanto fanno parte della palette di colori utilizzata. 
	\item (TABELLE?)
\end{itemize}

\subsection{Schema database}
Schema relazionale del database:



\section{Progettazione}
La progettazione è iniziata poco dopo l'inizio dell'assegnazione dei gruppi.
\noindent
\\
Si è rapidamente deciso che l'argomento del sito sarebbe stata una palestra. Dopo aver deciso il nome, abbiamo iniziato a stilare una lista di operazioni che avremmo potuto permettere agli utenti. Successivamente, ci siamo focalizzati sul contenuto e lo stile del site, nonché sulla struttura generale. È stato in quest'occasione che abbiamo deciso di introdurre dei cicli nella struttura gerarchica e le breadcrumb.
\noindent
\\
Abbiamo riscontrato problemi quando abbiamo dovuto aggiungere del comportamento al sito, in particolare nella parte di backend: nell'analisi dei requisiti non abbiamo incluso le funzionalità che avrebbero dovuto effettivamente essere presenti in quanto progetto, e questo ci si è ritorto contro verso la fine del progetto, che abbiamo dovuto aggiungere ulteriori funzionalità.
\noindent
\\
Ad ogni incontro sincrono abbiamo messo per iscritto quello che era stato fatto, quello che avremmo dovuto fare tra quell'incontro e quello successivo, e quali sarebbero stati gli argomenti da affrontare in seguito. Questo ci ha permesso di tenere un buon ritmo ed effettuare modifiche rapidamente quando ci siamo accorti de essere finiti fuori scope.


\section{Presentazione}
\subsection{CSS}
All'interno del nostro CSS vengono utilizzate delle variabili, che vengono impostate all'inizio del file e riutilizzate successivamente, grazie a questo riusciamo a tenere sotto controllo i vari contrasti e i colori utilizzati.
Vengono utilizzati tre diversi fogli di stile, ognuno con uno scopo diverso: 
\begin{itemize}
	\item style.css: per lo stile da desktop 
	\item mini.css: per lo stile da mobile 
	\item print.css: per lo stile da stampante
\end{itemize}
\noindent
È stata presa questa decisione è stata presa per gestire al meglio l'aspetto responsive del sito, oltre ad avere una propria organizzazione del css. 

In particolare, per quanto riguarda il css della stampa...
(TODO: AGGIUNGERE SPIEGAZIONE DI COM'è IL CSS DELLA STAMPA?)


\subsection{Immagini e icone}
La maggior parte delle immagini vengono archiviate prevalentemente in formato JPG, con una dimensione massima che non supera il MB, ad eccezzione di questo troviamo le immagini generate per le pagine error e le icon, che vengono salvate in formato WEBP, e l'immagine del logo che è salvata in formato PNG ma non supera la grandezza di 1MB. Questa scelta è stata adottata per ridurre i tempi di caricamento delle pagine; tuttavia, l’utilizzo di questo formato potrebbe comportare possibili problemi di retrocompatibilità, dovuti alla sua introduzione relativamente recente.


\section{Accessibilità}

\subsection{Palette dei colori e contrasti}
La palette dei colori è stata scelta per garantire un elevato contrasto tra testi e sfondi, rispettando le linee guida WCAG. Ad esempio, testi chiari su sfondi scuri per una migliore leggibilità, evitando combinazioni che potrebbero causare difficoltà a utenti con disabilità visive.

\bigskip
\noindent
\textbf{Palette dei colori utilizzata:}

\begin{center}
	\begin{tabular}{|c|c|c|}
		\hline
		\rowcolor{lightgray} \textbf{Colore} & \textbf{Codice HEX} & \textbf{Utilizzato per} \\
		\hline
		\cellcolor[HTML]{c0ebfb}\hspace{2cm} & \texttt{\#c0ebfb}  & Links non visitati \\
		\hline
		\cellcolor[HTML]{965cfd}\hspace{2cm} & \texttt{\#965cfd}  & Links non visitati \\
		\hline
		\cellcolor[HTML]{ffee55}\hspace{2cm} & \texttt{\#ffee55}  & Testo \\
		\hline
		\cellcolor[HTML]{ffcc00}\hspace{2cm} & \texttt{\#ffcc00}  & Testo \\
		\hline
		\cellcolor[HTML]{ee8800}\hspace{2cm} & \texttt{\#ee8800}  & Testo \\
		\hline
		\cellcolor[HTML]{333322}\hspace{2cm} & \texttt{\#333322} &  Sfondo per tabelle \\
		\hline 
		\cellcolor[HTML]{181211}\hspace{2cm} & \texttt{\#181211}  & Sfondo \\
		\hline
		\cellcolor[HTML]{111111}\hspace{2cm} & \texttt{\#111111}  & Sfondo header\\
		\hline
		\cellcolor[HTML]{dddddd}\hspace{2cm} & \texttt{\#dddddd}  & Testi secondari \\
		\hline
	\end{tabular}
\end{center}
\bigskip 
I colori principali utilizzati nel sito sono riportati nella tabella; eventuali altri colori, impiegati in modo occasionale, sono stati comunque scelti e verificati nel rispetto dei principi di accessibilità.


\subsection{Lingue straniere}
Il sito supporta contenuti in italiano come lingua principale, ma è progettato per essere facilmente adattabile a lingue straniere attraverso l'uso di attributi lang e meccanismi di internazionalizzazione, permettendo una migliore accessibilità per utenti non italofoni.

\subsection{Screen reader}
L’accessibilità per gli utenti di screen reader è stata migliorata attraverso l’uso appropriato della semantica HTML5 e degli attributi ARIA. Questo è stato reso possibile attraverso l'uno di screen reader come NVDA che ci hanno aiutato a simulare l'esperienza di navigazione di un utente che ne necessita il bisogno.

\paragraph{Aiuti alla navigazione}
All’inizio di ogni pagina è presente un collegamento di salto al contenuto principale (“Vai al contenuto”), posizionato come primo elemento del DOM. Questo link è inizialmente nascosto e diventa visibile quando riceve il focus da tastiera, consentendo agli utenti che navigano senza mouse di bypassare elementi ripetitivi come l’intestazione e il menu di navigazione e raggiungere direttamente il contenuto principale racchiuso nel tag \texttt{<main>}.

\paragraph{Breadcrumb}
Nessuno ama perdersi tra le pagine di un sito, quindi abbiamo aggiunto i breadcrumb per dare un riferimento chiaro a chi naviga. Non è solo estetica: abbiamo usato il tag \texttt{<nav>} con l'etichetta \texttt{aria-label="percorso di navigazione"} proprio per non creare confusione a chi usa uno screen reader, distinguendo nettamente questo blocco dal menu principale. Un dettaglio fondamentale è l'uso di \texttt{aria-current="page"} sull'ultimo elemento: serve a dire chiaramente "sei qui", confermando la posizione attuale senza margini d'errore.

\paragraph{Tabelle dati}
Le tabelle utilizzate per visualizzare gli orari dei corsi sono state progettate per essere facilmente interpretabili dagli screen reader. In particolare:
\begin{itemize}
	\item al posto del tag \texttt{<caption>}, è stato utilizzato un paragrafo descrittivo nascosto visivamente, collegato alla tabella tramite l’attributo \texttt{aria-describedby}. Questo consente allo screen reader di leggere una descrizione introduttiva del contenuto della tabella prima della navigazione delle celle;
	\item le celle di intestazione (\texttt{<th>}) utilizzano l’attributo \texttt{scope} per associare correttamente le celle di dati alle rispettive intestazioni di riga e di colonna, permettendo una lettura contestualizzata delle informazioni.
\end{itemize}

\paragraph{Menu degli abbonamenti}
Nella sezione dedicata agli abbonamenti, la selezione delle diverse categorie è implementata tramite un menu di navigazione semantico contenente pulsanti standard (\texttt{<button>}). L’utilizzo di elementi HTML nativi garantisce la corretta accessibilità da tastiera e una gestione automatica del focus, evitando l’uso di ruoli ARIA complessi e mantenendo il codice semplice e semanticamente corretto.

\paragraph{Gestione della lingua}
Per assicurare una corretta pronuncia da parte dei sintetizzatori vocali, i termini in lingua inglese presenti all’interno del testo italiano sono racchiusi in elementi \texttt{<span>} con attributo \texttt{lang="en"}.


\subsection{Compatibilità}
I contenuti di ogni pagina sono stati disposti in modo tale da essere facilmente accessibili indipendentemente dal dispositivo utilizzato.
Abbiamo adottato un menù ad hamburger con css puro per gli utenti da mobile, facilitandone la navigazione; le varie tabelle presenti inoltre sono state rese responsive in modo fa facilitarne la lettura su diversi dispositivi.

\section{TEST EFFETTUATI}

\subsection{Sicurezza}


\subsection{Accessibilità}


\section{Divisione dei compiti}
Vengono qui esposti i vari compiti svolti da ciascun membro del team di progetto: \\
\noindent
\\
\textbf{Armando Moda Scarati}
\begin{itemize}
	\item Creazione delle e gestione pagine: \texttt{abbonamenti.html}, \texttt{palestra.html},  \texttt{utente-admin.html},  \texttt{utente-pt.html},  \texttt{utente-semplice.html}
	\item Gestione in css delle stesse 
	\item conclusione e revisione della relazione
\end{itemize}
\textbf{Alessandro Morabito}
\begin{itemize}
	\item Creazione e gestione dei file:
	\begin{itemize}
		\item \texttt{cerca-istruttore.php} (con css mobile, dekstop e stampa)
		\item \texttt{profilo-istruttore.php} (con css mobile, dekstop e stampa)
		\item \texttt{home.php} (con css mobile, dekstop e stampa)
		\item \texttt{download-scheda.php}
		\item \texttt{upload-messaggio.php} (utilizzato solamente per dare un significato al form contattaci. I dati vengono in realtà salvati nel database)
		\item \texttt{lista-clienti-pt.php} i controlli sul file pdf (scheda da parte del personal trainer) caricato (grandezza posta a 1MiB in quanto possibilmente vincolato da configurazione php, controllo estensione file, controllo MIME file)
		\item \texttt{utente-semplice.php} pulsante \texttt{Visualizza PDF (...)} per la visualizzazione della scheda caricata da un personal trainer
		\item \texttt{caricaScheda.js}
		\item \texttt{caricaMessaggio.js}
		\item \texttt{cercaIstruttore.js}
		\item menù mobile (css puro)
		\item \texttt{.htaccess} (impostati riferimenti a errore 403 e 404, riscrittura condizionata URL: se si prova ad accedere alla \textit{root} del sito, viene aggiunto \texttt{/home.php} all'indirizzo)
	\end{itemize}
	\item Gestione in css delle stesse 
	\item conclusione e revisione della relazione
\end{itemize}
\textbf{Granziero Luca}
\begin{itemize}
	\item Creazione e gestione delle pagine: \texttt{corsi.html}  \texttt{salapesi.html} \texttt{error404.html}, \texttt{erro403.html}
	\item Gestione in css delle stesse
	\item Creazione delle pagine js: toggle-descrizione-orari.js 
	\item Stesura e aggiornamento della relazione 
\end{itemize}
\textbf{Marco}
\begin{itemize}
	\item Gestione della pagine: \texttt{abbonamenti.html}, \texttt{contattaci.html},  \texttt{login.html},  \texttt{registrati.html}
	\item Gestione in css delle stesse
	\item Crezione prima versione del database
	\item conclusione e revisione della relazione
\end{itemize}

\section{Conclusioni e sviluppi futuri}
Il progetto costituisce una base strutturata per lo sviluppo di un sito web per una palestra, progettato in modo da consentire agevolmente future espansioni e l’implementazione di ulteriori funzionalità, quali ad esempio:
\end{document}
